
\ocpbuild{} is a tool to build whole OCaml projects, composed of
multiple files in multiple directories. 

\section{Main Features}

The main features of \ocpbuild{} are:

\begin{itemize}
\item Simple declarative project description
\item Short syntax for packing modules, 
  dependencies between libraries.
\item Simple way to customize options for every module.
\item Incremental and parallel build system
\item Support for camlp4, composition of syntax extensions and
  optimised compilation
\item Portable, run both on Unices and Windows
\item Support and generate ocamlfind META files
\end{itemize}

\section{Comparing \ocpbuild{} to others}

Here is a list of other building tools for OCaml projects and how
\ocpbuild{} compares to them.

\begin{description}
\item[ocamlbuild:] ocamlbuild is a generic build manager, where
  configuration files are written in OCaml. \ocpbuild{} has a less
  generic scope, but consequently, its configuration files are much
  simpler, and still very powerful because of its specialization for
  OCaml projects. Also, \ocpbuild{} works under native Windows.
\item[make:] make is a well-known generic build manager for
  Unices. Makefiles for OCaml projects are notoriously complicated to
  write, and do not support parallelization well.
\item[omake:] omake is an improved make, written in OCaml, and with
  native support for OCaml projects. omake works well on big projects
  and under Windows, but it is not supported anymore, and omake has no
  support for library dependencies.
\item[ocamlfind] ocamlfind is not a build manager, but a command-line
  tool to make the OCaml compilers aware of other OCaml components
  installed in the system (libraries, syntaxes, etc.), thanks to their
  META files. \ocpbuild{} supports META files, and, since it reads the
  environment only once, is much faster than using ocamlfind for every
  module.
\item[oasis:] oasis is a simple frontend to describe OCaml projects,
  inspired from Cabal. Configuration files are simple declarative
  files, as with \ocpbuild{}, but the build process is done by other
  tools (ocamlbuild usually). \ocpbuild{} is very similar to using
  oasis+ocamlfind+ocamlbuild, but supports multi-components projects,
  and its declarative language is simpler.
\item[opam:] opam is a source package manager for OCaml. It relies on
  other build managers to compile packages, and only decides in which
  order the packages should be compiled and installed to meet
  dependencies between them. opam uses \ocpbuild{} to build itself.
\end{description}

